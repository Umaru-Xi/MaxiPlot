\documentclass[11pt,a4paper]{article}
\usepackage{graphicx}
\usepackage{amsmath}
\usepackage[amsmath]{maxiplot}
\usepackage{esint}

\usepackage[UTF8]{ctex}

\title{Maxiplot: 在\LaTeX 中使用 Maxima 和 Gnuplot in .\\}
\date{2022年01月31日}

\def\Maxima{\emph{Maxima}}
\def\Gnuplot{\emph{Gnuplot}}

\begin{document}
\maketitle

\section{简介}
众所周知, \Maxima 是一个可以被用来计算微积分, 解方程, 求极限, 计算矢量或矩阵, 创建图表以及许多许多事的符号计算程序. 经过功能拓展, 它甚至具有编写程序的功能(通过Lisp语言). 如果这些都不足以满足你的需求, 它是基于 GNU GPL 许可发行, 所以你可以自由的从 \texttt{http://maxima.sourceforge.net} 获取. 同时, 那里有许多不同语言编写的说明文档.

这个项目提供了一个 \LaTeX{} 宏包以提供直接通过'编程'向Tex文件导入结果的功能, 这个过程不需要再依靠其他文件和借口. Maxima 代码可以直接嵌入到\LaTeX{} 文档中. 当它在处理文档时, 会生成一个后缀名为 \texttt{.mac}的Maxima脚本文件. 这个新的文件可以直接被Maxima处理, 并自动创建另一个后缀名为 \texttt{.mxp}的文件供\LaTeX{} 做进一步处理. 

\Gnuplot{} 指令也可以被潜入到Tex文件中, 这归功于J. M. Mira的工作. 使用Gnuplot 将会引入另一个后缀名为 \texttt{.gnp} 辅助文件. Gnuplot将如同Maxima处理 \texttt{.mac} 文件一样处理它.

最新版本的Maxiplot是在2013年发布的, 今天许多程序都发生了变化. 起初得知Maxiplot时我是无比兴奋的, 但后来发现它无法在我的膝上计算机上正常工作. 这份分支代码修复了许多我遇到的问题, 并且添加了一些方便的自动化构建的脚本. 另外, 根据修改后的代码重新编写了帮助文件. (这个分支代码使用XeLaTex作为编译器)

\section{安装}

使用新的版本需要拷贝\texttt{maxiplot.sty}, \texttt{Makefile}, \texttt{build.sh} 或者 \texttt{build.csh}. 并且你需要预先安装 maxima, xelatex, bash 或者 csh. 如果你使用openSUSE Leap, 可以使用sudoers组的用户执行下面的指令进行安装:

\texttt{zypper ref}

\texttt{zypper ins texlive make maxima bash csh}

原始的项目只需要拷贝 \texttt{maxiplot.sty} 到\LaTeX{} 可以找到的路径, 或者与你的文档相同的目录下. 

\section{ \LaTeX{} 宏包 \texttt{maxiplot}}

\subsection{如何使用?}

要使用这个分支代码, 只需要使用下面的指令:

\texttt{bash build.sh mydocument.tex}

或使用C SHELL:

\texttt{csh build.csh mydocument.tex}

然后, 你就会在build目录下得到 \texttt{mydocument.pdf} 啦.

如果你也使用KDE Kile作为集成编辑器, 你可以参考下面的步骤将这份分支的构建脚本添加到编译工具中:

\texttt{1. 打开 '设置 - 配置 Kile - 工具 - 构建';}

\texttt{2. 在 '选择工具' 栏中点击 '新建', 键入一个像 'Make' 这样的名字, 选择类为 'XeLatex'. 然后点击完成;}

\texttt{3. 选择你添加的新工具, 在 '常规' 一栏中键入指令 'bash build.sh' 或 'csh build.csh'. 并在选项中键入 '\%source';}

\texttt{4. 最后一步, 设置 '高级' 一栏中 '类型' 为 '在Konsole中运行' , '类' 为 '编译'. 现在, 你将可以在构建选项卡中找到它.}

{这份分支的脚本将会刷新你的Tex文件, 这会使得Kile无法自动重载. 但在一般情况下无需担心.}

以下是原始分支的构建方法:

用法非常简单, 像平时一样处理完你的文档. 然后运行下面的指令:

\texttt{latex mydocument.tex}

你将看到 \texttt{mydocument.mac} 出现在了你的工作目录下. 用 Maxima 处理它:

\texttt{maxima -b mydocument.mac}

如果你用到了 \Gnuplot{} 指令:

\texttt{gnuplot mydocument.gnp}

重新处理 \LaTeX{} 文档, \textit{et voil\`a!} (给你吧!).

如果你的发行版允许, 你可以使能\texttt{write18} 指令来允许 \Maxima{} 和 \Gnuplot{} 在你处理 \LaTeX{} 文档时自动运行 (你需要提前添加可执行文件的安装目录到系统). 

\subsection{用户接口}
\subsubsection{Maxima.}

这一节和下一节将使用一些例子来展示\texttt{maxiplot} 的用法. 这需要一些Maxima 的基础知识.

这个宏包有一个选项来允许兼容\texttt{amsmath}提供的 \texttt{pmatrix} 环境. 因此, 如果你需要用该环境创建矩阵, 可以在你的文档中添加下面的内容:
\begin{verbatim}
\usepackage{amsmath}
\usepackage[amsmath]{maxiplot}
\end{verbatim}

最重要的环境是\texttt{maxima} 和\texttt{maximacmd}.他们提供的内容将用语产生 \texttt{.mac} 文件以供Maxima使用. 因此, 这些环境中应该使用非\LaTeX 风格的注释符号\% . 也就是说, \% 不能在Maxima代码中用于注释.  你应该使用C语言风格的语法(/* \textit{注释} */). 代码将被以参数的形式传输给处理函数, 因此, 他们应该以逗号分隔.

这是一个简单的例子:
\begin{verbatim}
\[   %进入数学编辑模式
\begin{maxima}
  f: x/(x^3-3*x+2),     /* 求积分 */
  tex('integrate(f,x)), /* 这里输出积分表达式... */
  print("="),
  tex(integrate(f,x)),  /* ...这里是结果 */
  print("+K")
\end{maxima}
\]   %离开数学编辑模式
\end{verbatim}

如果使用本项目处理这段代码, 将会得到:

\[
\begin{maxima}
  f: x/(x^3-3*x+2),     
  tex('integrate(f,x)), 
  print("="),
  tex(integrate(f,x)),  
  print("+K")
\end{maxima}
\]
\vskip1em
有一些\texttt{maxima} 不能被包含的情况. 这时可以使用能马上给出结果的\texttt{maxima*}{} . 它的输出可以在稍后通过 \verb|\maximacurrent| 指令插入, 像这样:
\begin{verbatim}
\begin{maxima*}
  suml(L):=lsum(i,i,L), 
  printrow(L):=block(
    [str:""],
    for i:1 step 1 thru length(L)-1 do(
        str:concat(str,L[i],"&")),
    str:concat(str,L[length(L)],"\\\\"),
    print(str)),
  xi:[1,2,3,4,5,6],
  fi:[3,4,7,10,8,2],
  for i:1 while i<=length(xi) do (
    printrow([xi[i],fi[i],(fi*xi)[i],(fi*xi^2)[i]])
    ),
  print("\\hline"),
  printrow(["",N:suml(fi),fx:suml(fi*xi),fx2:suml(fi*xi^2)])
\end{maxima*}
                 
\begin{center}
  \begin{tabular}{|c|c|c|c|c|}
  $x_i$&$n_i$&$n_i\cdot x_i$&$n_i\cdot x_i^2$\\
  \hline
  \maximacurrent
  \end{tabular}
\end{center}
\end{verbatim}

\begin{maxima*}
  suml(L):=lsum(i,i,L), 
  printrow(L):=block(
    [str:""],
    for i:1 step 1 thru length(L)-1 do(
        str:concat(str,L[i],"&")),
    str:concat(str,L[length(L)],"\\\\"),
    print(str)),
  xi:[1,2,3,4,5,6],
  fi:[3,4,7,10,8,2],
  for i:1 while i<=length(xi) do (
    printrow([xi[i],fi[i],(fi*xi)[i],(fi*xi^2)[i]])
    ),
  print("\\hline"),
  printrow(["",N:suml(fi),fx:suml(fi*xi),fx2:suml(fi*xi^2)])
\end{maxima*}
                 
\begin{center}
  \begin{tabular}{|c|c|c|c|c|}
  $x_i$&$n_i$&$n_i\cdot x_i$&$n_i\cdot x_i^2$\\
  \hline
  \maximacurrent
  \end{tabular}
\end{center}

需要留心 \verb|\maximacurrent| 指令将替换\emph{最后一个 \texttt{maxima}
  代码块} 的输出结果, 所以必须在所有 \texttt{maxima} 代码块之前使用.

  如果你想要稍后使用计算结果或者你想到别处使用它们, 你可以增加一些选项来存储它们. 上面的例子可以通过以下的方式重新实现:
\begin{verbatim}
\begin{maxima*}[table]
  suml(L):=lsum(i,i,L), 
  printrow(L):=block(
    [str:""],
    for i:1 step 1 thru length(L)-1 do(
        str:concat(str,L[i],"&")),
    str:concat(str,L[length(L)],"\\\\"),
    print(str)),
  xi:[1,2,3,4,5,6],
  fi:[3,4,7,10,8,2],
  for i:1 while i<=length(xi) do (
    printrow([xi[i],fi[i],(fi*xi)[i],(fi*xi^2)[i]])
    ),
  print("\\hline"),
  printrow(["",N:suml(fi),fx:suml(fi*xi),fx2:suml(fi*xi^2)])
\end{maxima*}
                 
\begin{center}
  \begin{tabular}{|c|c|c|c|c|}
  $x_i$&$n_i$&$n_i\cdot x_i$&$n_i\cdot x_i^2$\\
  \hline
  \table
  \end{tabular}
\end{center}
\end{verbatim}

注意, 当输入 ``\texttt{table}'' 作为\texttt{maxima*} 的参数时 \emph{反斜杠} (\verb|\|) \emph{是不需要的}.

这是编辑模式下具有同样用途的例子:  \verb|\imaxima| 指令(命名来自
``inline maxima'' , 嵌入的Maxima).
\begin{verbatim}
  \[
  \overline{x}=\imaxima{tex(xx:fx/N)}\qquad
  \sigma^2=\imaxima{tex(sx2:fx2/N-xx^2)}\qquad
  \sigma=\imaxima{tex(sqrt(sx2))}
  \]       
\end{verbatim}

\[ 
\overline{x}=\imaxima{tex(xx:fx/N)}\qquad
\sigma^2=\imaxima{tex(sx2:fx2/N-xx^2)}\qquad
\sigma=\imaxima{tex(sqrt(sx2))}
\]

当没有输出文件时(如定义函数或加载Maxima库),  应该使用\texttt{maximacmd} 环境或\verb|\imaximacmd| 指令. 他俩没有\texttt{*} 变体或任何其他选项. 进一步的, \Maxima{} 内置指令必须通过分号 (;) 分开. 如果可以, 用美元符号(\$)则更好. 让我们来看一下 \Maxima/\Gnuplot{} 接口的例子. 这个例子展示了如何绘制 $\sin$ 函数和它在$\frac{\pi}{3}$处的切线:

\begin{verbatim}
\begin{maximacmd}
  tangent(fx,a):=expand(ev(fx,x=a)
                  +subst(a,x,diff(fx,x))*(x-a))$
  plot2d([sin(x),tangent(sin(x),%pi/3)], [x,-3,3],
         [gnuplot_preamble,"set zeroaxis;"],
         [gnuplot_term, png],
         [gnuplot_out_file,"./\jobname2D.png"])$
\end{maximacmd}
\begin{center}
  \mxpIncludegraphics[scale=0.60]{\jobname2D.png}
\end{center}
\end{verbatim}   

这个代码创建了一个\verb|png| 格式的图像文件 \texttt{\jobname2D.png}:
\begin{maximacmd}
  tangent(fx,a):=expand(ev(fx,x=a)
                  +subst(a,x,diff(fx,x))*(x-a))$
  plot2d([sin(x),tangent(sin(x),%pi/3)], [x,-3,3],
         [gnuplot_preamble,"set zeroaxis;"],
         [gnuplot_term, png],
         [png_file,"./\jobname2D.png"])$
\end{maximacmd}
\begin{center}
  \mxpIncludegraphics[scale=0.60]{\jobname2D.png}
\end{center}



这个环境包含的 \LaTeX{} 命令在加入\texttt{mac}文件之前将被替换. 有时遇到具体的字符串可能会引入麻烦, 所以这是可能并不需要这个功能. \texttt{vmaxima} 和\texttt{vmaximacmd} 环境可以解决这个问题. 它们的用法与之前一样, 但是具体的文本将被直接传送. 本环境就基于\texttt{verbatim} \LaTeX{} 宏包.

\pagebreak

\subsubsection{Gnuplot}
 \Maxima{} 通过 \Gnuplot{} 创建图表, 有时我们需要直接使用后者. 这时, 可以使用 \texttt{gnuplot} 和它的详细版本 \texttt{vgnuplot} .

这是一个绘制3维图像的例子:
\begin{verbatim}
\begin{gnuplot}
  set term png crop enhanced font "calibri, 10"
  set output "toros.png"
  set parametric
  set urange [0:2*pi]
  set vrange [-pi:pi]
  set isosamples 36,24
  set hidden3d
  set view 75,15,1,1
  unset key
  set ticslevel 0
  x1(u,v)=cos(u)+.5*cos(u)*cos(v)
  y1(u,v)=sin(u)+.5*sin(u)*cos(v)
  z1(u,v)=.5*sin(v)
  x2(u,v)=1+cos(u)+.5*cos(u)*cos(v)
  y2(u,v)=.5*sin(v)
  z2(u,v)=sin(u)+.5*sin(u)*cos(v)
  set multiplot
  splot x1(u,v), y1(u,v), z1(u,v) w pm3d, x2(u,v), y2(u,v), z2(u,v) w pm3d
  splot x1(u,v), y1(u,v), z1(u,v) lt 3,   x2(u,v), y2(u,v), z2(u,v) lt 5 
\end{gnuplot}
\begin{center}
  \mxpIncludegraphics[scale=0.75]{toros.png}
\end{center}
\end{verbatim}


\begin{gnuplot}
  set term png crop enhanced font "calibri, 10"
  set output "toros.png"
  set parametric
  set urange [0:2*pi]
  set vrange [-pi:pi]
  set isosamples 36,24
  set hidden3d
  set view 75,15,1,1
  unset key
  set ticslevel 0
  x1(u,v)=cos(u)+.5*cos(u)*cos(v)
  y1(u,v)=sin(u)+.5*sin(u)*cos(v)
  z1(u,v)=.5*sin(v)
  x2(u,v)=1+cos(u)+.5*cos(u)*cos(v)
  y2(u,v)=.5*sin(v)
  z2(u,v)=sin(u)+.5*sin(u)*cos(v)
  set multiplot
  splot x1(u,v), y1(u,v), z1(u,v) w pm3d, x2(u,v), y2(u,v), z2(u,v) w pm3d
  splot x1(u,v), y1(u,v), z1(u,v) lt 3,   x2(u,v), y2(u,v), z2(u,v) lt 5 
\end{gnuplot}
\begin{center}
  \mxpIncludegraphics[scale=0.75]{toros.png}
\end{center}

让我们来测试一下 \verb|\mxpIncludegraphics| 指令: 它的用法与\verb|graphicx|包中的 \verb|includegraphics| 相同; 事实上, 他仅仅是在引用宏之前调用已经存在的绘图文件.

\subsection{问题}
这是一个专业的版本, 许多Maxima功能没有经过测试, 并且也没有与更多的\LaTeX{} 包共同测试. 因此, 它一定需要调整.

然而, 我想大多数问题都将会有详细的输出. 如果计算结果表达式过长将不容易被分割成多行(当然, 你可以在 Maxima 中处理好并且粘贴到文档中).

问题也可能是\LaTeX{} 文档引起的. 一般而言, Maxima反序处理字符输入. 如果我们输入:\\
\verb|  $$\imaxima{tex(x+y+z+t=0)}$$|\\
事实上将得到:
$$\imaxima{tex(x+y+z+t=0)}$$

可以通过Maxima函数 \texttt{ordergreat} 和\texttt{unorder}解决:
\begin{verbatim}
\imaximacmd{ordergreat(x,y,z,t)$}
$$\imaxima{tex(x+y+z+t=0)}$$
\imaximacmd{unorder()$}
\end{verbatim}
%\imaximacmd{ordergreat(x,y,z,t)$}
%$$\imaxima{tex(x+y+z+t=0)}$$
%\imaximacmd{unorder()$}

进一步的, 如果希望对齐多个方程, 我们需要一些跟进一步的知识(内嵌Lisp语言):
\begin{verbatim}
\begin{maximacmd}
  ordergreat(x,y,z)$
  :lisp(defprop mequal (&=) texsym)
\end{maximacmd}

\begin{maxima*}
  eq1:a-2*b=x+y,
  eq2:b=2*x-3*y+2*z,
  tex(eq1),
  print("\\\\"),
  tex(eq2)
\end{maxima*}

\begin{maximacmd}
  unorder()$
  :lisp(defprop mequal (=) texsym)    
\end{maximacmd}
\end{verbatim}

\begin{maximacmd}
  ordergreat(x,y,z)$
  :lisp(defprop mequal (&=) texsym)
\end{maximacmd}

\begin{maxima*}
  eq1:a-2*b=x+y,
  eq2:b=2*x-3*y+2*z,
  tex(eq1),
  print("\\\\"),
  tex(eq2)
\end{maxima*}

\begin{maximacmd}
  unorder()$
  :lisp(defprop mequal (=) texsym)    
\end{maximacmd}

\begin{align}
  \maximacurrent
\end{align}

\section{最后一点话}
就像我之前提到的, 这是一个专业的宏包, 它需要进一步的修改. 所以欢迎任何想法和评论.
\\

\raggedleft{Jos\'e Miguel M. Planas}\\
\raggedleft{$<$nohaim@gmail.com$>$}\\[10pt]
\raggedleft{(英语翻译由 Jaime Villate 提供)}\\
\raggedleft{(中文翻译由 Umaru Aya(凝萌$\cdot $曦子) 提供)}

\raggedright

\section{关于这个分支}
这个文档实在原始文档的基础上直接修改的, 所以你在阅读时应该多加小心. 新修改的内容只在我的膝上计算机上测试通过. 同样欢迎任何想法和评论.
\\
\raggedleft{凝萌$\cdot $曦子}\\
\raggedleft{$<$umaru@umaru.science$>$}\\[10pt]
                  
\end{document}
